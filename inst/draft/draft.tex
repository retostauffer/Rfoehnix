\documentclass[article,nojss,shortnames]{jss}

\usepackage{thumbpdf,lmodern}
\usepackage{framed}
\usepackage{algorithmic}
\usepackage{amsmath}
\usepackage[]{algorithm2e}


% Draft
%\hypersetup{draft}
%\shortcites{rasmussen2012}

\author{Reto Stauffer\\Universit\"at Innsbruck
\And Matthias Dusch\\Universit\"at Innsbruck
\AND Fabien Maussion\\Universit\"at Innsbruck
\And Georg J. Mayr\\Universit\"at Innsbruck}

\Plainauthor{Reto Stauffer, Matthias Dusch, Fabien Maussion, Georg J. Mayr}

\title{phoeton: Software for Objective Probabilistic Foehn Classification}
\Plaintitle{phoeton: Objective Probabilistic Foehn Classification}
\Shorttitle{phoeton: Objective Probabilistic Foehn Classification}

\Abstract{
    This is the abstract here \dots
}

\Keywords{meteorology, foehn, objective classification, flexible mixture model}
\Plainkeywords{meteorology, foehn, objective classification, flexible mixture model}

\Address{
  Reto Stauffer\\
  Department of Statistics\\
  Faculty of Economics and Statistics\\
  Universit\"at Innsbruck\\
  Universit\"atsstr.~15\\
  6020 Innsbruck, Austria\\
  E-mail: \email{Reto.Stauffer@uibk.ac.at}\\
  URL: \url{https://retostauffer.org}\\

  Matthias Dusch, Fabien Maussion, Georg J. Mayr\\
  Department of Atmospheric and Cryospheric Sciences\\
  Faculty of Geo- and Atmospheric Sciences\\
  Universit\"at Innsbruck\\
  Innrain~52f\\
  6020 Innsbruck, Austria\\
  E-mail: \email{Georg.Mayr@uibk.ac.at},
          \email{Matthias.Dusch@uibk.ac.at},
          \email{Fabien.Maussion@uibk.ac.at}
}


\begin{document}

% -------------------------------------------------------------------
% SECTION: INTRODUCTION
% -------------------------------------------------------------------
\section{Introduction}

Introduction why this is so great.





% -------------------------------------------------------------------
% SECTION: METHODS
% -------------------------------------------------------------------
\section{Methods}

\textit{Note by Reto:}
The original publication (\code{FlexMix}, \citealt{gruen2008})
provides general framework for flexible mixture models for multiple
clusters. For this application we will concentrate on the special
case with only two distinct clusters ($K = 2$).

\subsection{Finite Mixture Models with Two Clusters}

The density $h(\dots)$ of a finite mixture model can be written as follows:

\begin{equation}
    h(\mathit{y} ~|~ \mathbf{x}, \mathit{\omega}, \mathit{\alpha}, \mathit{\theta}) =
        \sum_{k=1}^K \pi_k(\mathbf{\omega}, \mathit{\alpha}) \cdot f_k(\mathit{y} ~|~ \mathbf{x}, \mathit{\theta}_k)l
\end{equation}

The density is the sum over the individual densities $f(\dots)$ of the $K$ clusters
times the 


.....
....

$\mathit{y}$ denotes the response, $\mathbf{x}$ the predictors for the main part,
and $\mathit{\omega}$ the concomitant variables for the probability model.



... where in our case $K$ is 2 (two groups). $y$ denotes our response, $x$ the predictors
for the main part of the model, and $\omega$ the concomitant variables used for the
probability model.
The density $h(\dots)$
$f_k$ is the density of the (Gaussian) mixture model is the sum over the density
of the two groups $k=1,2$ multiplied by the probability $\pi_k$.

In case of a two-group Gaussian mixture model the density $f_\bullet$ is simply the
Gaussian density $\phi$ defined by the parameters $\theta_k = (\mu_k, \sigma_k)$ and
the probability of an observation being observed in $k=2$ is $1 - \pi_k$.
The only constraint is that $\sum_{k=1}^K = 1$, but basically any probability model
or classifier could be used. We are, for now, using a logistic regression model
with $g(\pi) = \omega^\top \alpha$. $\alpha$ are the regression coefficients from
the logistic regression model.



\subsection{Model Estimation}

To be able to estimate the model an maximum likelihood based EM algorithm is required.
It is assumed that there is an unobservable latent variable $z_n \in \{0, 1\}^K$
which exists for each observation $n = 1, \dots, N$ which defines the membership (the probability
that observation $n$ is a member of class $k$ (e.g., $z_{ik} = 1$ if observation $i$ comes from
class $k$).

Initially we have to guess $z_{nk}$.

$z_{i1} = \begin{cases} 0 & \text{if} ~~ y_i < \text{median}(y) \\ 1 & \text{else}\end{cases}$

In my case I am taking the first and third quartile
of `ff` for $\mu_1$, $\mu_2$ and the standard deviation of the data `ff` for the two
parameters $\sigma_1$, $\sigma_2$. Thus,

$\theta = (\theta_1, \theta_2) = (F^{-1}(y|p = 0.25), \text{sd}(y), F^{-1}(y|p = 0.75), \text{sd}(y))$

For the initial parameters $\alpha$ of the logistic model we can
apply a logistic regression model of the form:

$z = \frac{\exp(\omega^\top \alpha)}{1 + \exp(\omega^\top \alpha)}$

The following lines prepare the prior information $z$, $\theta$, and $\alpha$:





% -------------------------------------------------------------------
% SECTION: SOFTWARE
% -------------------------------------------------------------------
\section{Software}

Explain software.


% -------------------------------------------------------------------
% SECTION: DATA
% -------------------------------------------------------------------
\section{Data}

Maybe just combine data and the case study.

% -------------------------------------------------------------------
% SECTION: RESULTS
% -------------------------------------------------------------------
\section{Case Study/Results}


% -------------------------------------------------------------------
% SECTION: Discussion and Outlook
% -------------------------------------------------------------------
\section{Discussion and Outlook}


% -------------------------------------------------------------------
% Copernicus extra sections
% -------------------------------------------------------------------


\section*{Acknowledgments}

Acknowledgments dow here \dots


% -------------------------------------------------------------------
% Bibliography here
% -------------------------------------------------------------------
\newpage
\bibliography{references}

% -------------------------------------------------------------------
% Appendix section
% -------------------------------------------------------------------
\newpage

\begin{appendix}

\section{Appendix Section}

This is the appendix, if needed \dots

\end{appendix}

\end{document}

